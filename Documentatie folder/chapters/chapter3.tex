\chapter{Dafny}

\section{Limbajul de programare Dafny}
Dafny este un limbaj de programare funcțional și imperativ care prin intermediul adnotarilor sale permite crearea unui program care să nu conțină erori la rulare și să facă ceea ce își dorește programatorul.\par
Un exemplu de adnotare este $ requires \ \ a >= 10 $. Astfel, suntem siguri că variabila $a$ nu va fi mai mică decât 10 scutindu-ne de o structură de decizie pe care ar fi trebui să o utilizăm pentru a verifica acest lucru. Alte erori care pot fi evitate prin folosirea adnioarilor sunt erorile de indexare, împărțirea la 0, valori nule etc.
\par 


\section{Avantajele folosirii limbajului Dafny}
Precondițiile și postcondițiile din acest limbaj de programare stabilesc ce condiții trebuie să fie adevărate la intrare, respectiv, la ieșirea din metodă. Astfel, programele în Dafny sunt verificate pentru corectitudinea globală astfel încât fiecare rulare se va termina și se va ajunge la rezultatul dorit.
În plus, adăugarea acestor condiții duc la o înțelegere în detaliu a codului. 
%Dafny permite dovezi aserționale de corectitudine scrise că parte a textului programului.Dovezile de corectitudine explică motivele corectitudinii programului. În timp ce sunt furnizate ingrediente dovadă de către utilizator, etapele de verificare în sine sunt efectuate automat de către verificator.y. 




