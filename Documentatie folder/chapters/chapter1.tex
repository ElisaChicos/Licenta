\chapter{Paradigma Greedy}
Greedy este o strategie de rezolvare a problemelor de optimizare. Metoda presupune luarea unei decizii definitive la fiecare pas în funcție de informațiile cunoscute în prezent fără a ne îngrijora de efectul acesteia în viitor.


\section{Ce este o problemă de optimizare și cum lucrează metodă Greedy cu aceasta?}

O problemă de optimizare este o problema care are un input oarecare dar output-ul ei trebuie să fie o valoare maximă sau minimă.
În funcție de ce ne cere problema, această poate fi de două feluri: 
\begin{enumerate}
	\item Problema de minimizare 
	\par
	\fbox{\parbox{5.5in}{// de revenit }}
	\par
	%Scopul acestei probleme este de a găși valoarea care rezolvă problema și are cost minim
	\item Problema de maximizare
	\par
	\fbox{\parbox{5.5in}{// de revenit}}
\end{enumerate}

În funcție de decizia făcută de metoda Greedy, vom crea soluții posibile. O soluție posibilă este o submultime care satisface cerințele problemei, în timp ce o soluție optimă respectă cerințele date și are cost minim sau câștig maxim.
\par 
Dacă mai multe soluții îndeplinesc criteriile date, atunci acele soluții vor fi considerate ca fiind posibile, în timp ce soluția optimă este cea mai bună dintre toate soluțiile.
\par 
Drept urmare, această metodă este folosită pentru a determina
soluția oprimă și a rezolva corect problema de optimizare.


\section{Algoritmi Greedy}
\subsection{Problema Selecției Activităților}	
Problema activităților este o problemă de maximizare care presupune alegerea cât mai multor activități care pot fi realizate într-un anumit interval de timp fără că acestea să se suprapună.
\vspace{0.5cm}
\par	 
O posibilă formulare a acestei probleme este: \par
\fbox{\parbox{5.5in}
	{
		\setlength{\parindent}{0em}
		Input: \par
		\setlength{\parindent}{2em}
		$\bullet$ n - numărul de activități,
		\par 
		
		$\bullet$ aStart[0..n-1] - vector care conține timpul la care încep activitățile, 
		\par
		$\bullet$ aSfarsit[0..n-1] - vector ce conține timpul la care se termină actvitatile \par astfel încât aStart[i] $<$ aSfarsit[i] pentru orice 0 $\leq$ i $\leq$ n-1
		\setlength{\parindent}{0em}
		\par
		Output:\par
		\setlength{\parindent}{2em}
		$\bullet$ A $\subseteq$ $\{$ 0, ..., n-1 $\}$, unde A este o mulțime de activități care nu se\par suprapun și este de cardinal maxim;
}}

\par
\vspace{1cm}
Spre exmplu putem avea această mulțime de activități reprezentate în tabelul de mai jos:
\begin{table}[h!]
	\centering
	\begin{tabular}{||c c c c||} 
		\hline
		Nr & Activitate & Început & Final \\ [1ex] 
		\hline\hline
		1 & Curs ML & 8 & 10\\ \hline
		2 & Seminar AI & 10 & 12 \\\hline
		3 & Curs Pian & 11 & 1 \\\hline
		4 & Antrenament Înot & 12 & 17 \\ \hline
		5 & Curs CN & 16 & 18 \\\hline
		6 & Consultații RPA & 17 & 19 \\ \hline
		7 & Laborator CN & 18 & 20 \\ [1ex] 
		\hline
	\end{tabular}
\end{table}
\par
Soluția optimă este $A = \{1, 2, 4, 6  \} $


\subsection{Problema Codurilor Huffman}
\subsection{Problema Bin-packing}
\subsection{Algoritmul Dijkstra}
\subsection{Problema arborelui parțial de cost minim}

\section{Avantaje și dezavantaje ale algoritmilor Greedy}				
