\chapter{Paradigma Greedy}
Greedy este o strategie de rezolvare a problemelor de optimizare. Metoda presupune luarea unei decizii definitive la fiecare pas în funcție de informațiile cunoscute în prezent fără a ne îngrijora de efectul acesteia în viitor.


\section{Ce este o problemă de optimizare și cum lucrează \\ metoda Greedy cu aceasta?}

O problemă de optimizare este o problema care are un input oarecare dar output-ul ei trebuie să fie o valoare maximă sau minimă.
În funcție de ce ne cere problema, această poate fi de două feluri: 
\begin{enumerate}
	\item Problema de minimizare $\rightarrow$ rezultatul trebuie să fie de cost minim
	\par
	\fbox{\parbox{5.5in}{
	\setlength{\parindent}{0em}
	Input:\par
	\setlength{\parindent}{2em}
	$\bullet$ G = (V,E) - graf hamiltonial care are pe fiecare muchie un cost
	\par 
	\setlength{\parindent}{0em}
	Output:\par
	\setlength{\parindent}{2em}
	$\bullet$ A $\subseteq \{0,...m\}$, unde $ 0 < m <= |E|$ iar A este o multime de muchii a caror suma este minima. 
 }}
	\par 
	\setlength{\parindent}{2em}
	În problema de mai sus, avem ca input avem un graf hamiltonian în care fiecare muchie $E$ are un anumit cost. Problema cere să se returneze o mulțime A care conține muchii ale grafului ce formează un ciclu de cost minim.
	\par
	\vspace{4cm}
	\setlength{\parindent}{0em}
	\item Problema de maximizare $\rightarrow$ rezultatul trebuie sa fie de câștig maxim
	\par
	\fbox{\parbox{5.5in}{
	\setlength{\parindent}{0em}
	Input:\par
	\setlength{\parindent}{2em}
	$\bullet$ G = (V,E) - graf neorientat care are pe fiecare muchie un cost \par 
	$\bullet$ a - nod al grafului G \par 
	$\bullet$ b - nod al grafului G \par
	\par 
	\setlength{\parindent}{0em}
	Output:\par
	\setlength{\parindent}{2em}
	$\bullet$ A $\subseteq \{0,...m\}$, unde $ 0 <= m <= |E|$ iar A este o multime de muchii a caror suma este minima. 
}}
\par 
\setlength{\parindent}{2em}
Problema de mai sus primește ca input un graf în care fiecare muchie $E$ are un câștig și două noduri $a$ și $b$ și se cere să se găsească un drum între cele două noduri care să aibă câștig maxim.


\end{enumerate}

\vspace{0.5cm}
În funcție de decizia făcută de metoda Greedy, vom crea soluții posibile. O soluție posibilă este o submultime care satisface cerințele problemei, în timp ce o soluție optimă respectă cerințele date și are cost minim sau câștig maxim.
\par 
Dacă mai multe soluții îndeplinesc criteriile date, atunci acele soluții vor fi considerate ca fiind posibile, în timp ce soluția optimă este cea mai bună dintre toate soluțiile.
\par 
Drept urmare, această metodă este folosită pentru a determina
soluția optimă și a rezolva corect problema de optimizare.


\section{Algoritmi Greedy}
\subsection{Problema Selecției Activităților}	
Problema activităților este o problemă de maximizare care presupune alegerea cât mai multor activități care pot fi realizate într-un anumit interval de timp fără că acestea să se suprapună.
\vspace{0.5cm}
\par	 
O posibilă formulare a acestei probleme este: \par
\fbox{\parbox{5.5in}
	{
		\setlength{\parindent}{0em}
		Input: \par
		\setlength{\parindent}{2em}
		$\bullet$ n - numărul de activități posibile
		\par 
		
		$\bullet$ aStart[] - vector de lungime n care conține timpul la care \par încep activitățile, 
		\par
		$\bullet$ aSfarsit[] - vector de lungime n ce conține timpul la care se termină \par activitatile astfel încât aStart[i] $<$ aSfarsit[i] pentru orice 0 $\leq$ i $\leq$ n-1
		\setlength{\parindent}{0em}
		\par
		Output:\par
		\setlength{\parindent}{2em}
		$\bullet$ A $\subseteq$ $\{$ 0, ..., m $\}$, unde A este o mulțime de cardinal m de activități \par care nu se suprapun iar m are valoare maxima;
}}
\par
\vspace{1cm}
Spre exmplu putem avea această mulțime de activități reprezentate în tabelul de mai jos:
\begin{table}[h!]
	\centering
	\begin{tabular}{||c c c c||} 
		\hline
		Nr & Activitate & Început & Final \\ [1ex] 
		\hline\hline
		1 & Curs ML & 8 & 10\\ \hline
		2 & Seminar AI & 10 & 12 \\\hline
		3 & Curs Pian & 11 & 1 \\\hline
		4 & Antrenament Înot & 12 & 17 \\ \hline
		5 & Curs CN & 16 & 18 \\\hline
		6 & Consultații RPA & 17 & 19 \\ \hline
		7 & Laborator CN & 18 & 20 \\ [1ex] 
		\hline
	\end{tabular}
\end{table}
\par
Soluția optimă este $A = \{1, 2, 4, 6  \} $.
\par 
Strategia Greedy folosită pentru a ajunge la o soluție de câștig maxim este să alegem activitatea care se termină cel mai devreme, astfel ramanandu-ne mai mult timp pentru alte activități. \par 
Pentru exemplul de mai sus, la primul pas,  activitatea care se termină cel mai devreme este activitatea 1, deci vom participa la această și o vom adăuga la soluția finală.  La pasul următor, vom căuta activitatea care respectă aceeași proprietate dar trebuie să nu se suprapună cu nicio altă activitate la care am decis să participăm iar această va fi activitatea 2. Algoritmul va decurge în acest mod până când activitățile rămase nu vor putea fi alese deoarece se suprapun celor care se află în soluția finală.

\subsection{Problema Codurilor Huffman}

\section{Avantaje și dezavantaje ale algoritmilor Greedy}				
Avantajele pe care le prezintă algoritmii Greedy sunt date de simplitatea și intuitivitatea acestora, făcându-i foarte ușor de înțeles și totodată complexitatea în timp a acestora să fie mică. \par
Pe de altă parte, demonstrarea corectitudinii soluției generate de algoritm este greu de realizat deoarece pot există cazuri când soluția optimă la pasul curent nu va conduce la o soluție optimă la final.



