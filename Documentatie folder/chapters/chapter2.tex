\chapter{Problema Bancnotelor}

\section{Ce este Problema Bancnotelor?}

Problema bancnotelor este o problema de minimizare de care ne lovim zilnic. Stiind ca avem o  multime de bancnote $ B = \{ b_1, b_2, b_3\}$ si o suma $S$ de platit, rezultatul problemei va fi multimea de bancnote cu cardinal minim pe care o putem folosi pentru a plati suma $S$.

\subsection{Formularea problemei}

In limbaj natural, Problema Bancnotelor poate fi formulata astfel:
\par
\vspace{0.5cm}
\fbox{\parbox{5.5in}
	{ Se da o multime de bancnote $ B = \{ b_1, b_2, b_3\}$ si o suma $S$ pe care trebuie sa o platim. Se cere sa se afiseze o multime de bancnote care trebuie sa indeplineasca urmatoarele conditii:\par
	\setlength{\parindent}{1em}
	$\bullet$ cardinalul multimii sa fie minim, \par
	$\bullet$ suma elementelor multimii sa fie egata cu $S$;
}}
\par
\vspace{0.7cm}
Formularea computationala a problemei:\par
\vspace{0.5cm}
\fbox{\parbox{5.5in}
	{
		\setlength{\parindent}{0em}
		Input: \par
		\setlength{\parindent}{2em}
		$\bullet$ un numar natural n - suma care trebuie platita,
		\par 
		\setlength{\parindent}{0em}
		\par
		Output:\par
		\setlength{\parindent}{2em}
		$\bullet$ numerele $n_{500}, n_{200}, n_{100}, n_{50}, n_{20}, n_{10}, n_5, n_1$ ($n_i$ - numarul de \par bancnote i folosite), astfel incat $ \sum_{i \in \{500,200,100,50,20,10,5,1 \}} n_i$ sa fie minima \par si $ n =  \sum_{i \in \{500,200,100,50,20,10,5,1 \} } i \times n_i$.
}}
\par 
Spre exemplu, fiind date bancnotele $B = \{100,50,10,5,1\}$ si  $s = 157$ suma care trebuie platita. \par
Solutia optima pentru aceasta formulare a problemei este 
$\{1,1,0,1,2\}$ deoarece $ s = 1\cdot 100+1	\cdot 50+0	\cdot10,1	\cdot5+1	\cdot2 $.


\section{Strategia Greedy aleasa}

Strategia Greedy consta in alegerea, la fiecare pas, a bancnotei celei mai mari, care este mai mica sau egala cu suma pe care trebuie sa o platim. In functie de alegerea facuta, va trebui sa scadem valoarea bancnotei alese din suma care trebuie platita pentru a ne asigura ca avansam progresiv catre obiectivul final.
\par
Folosind exemplul de mai sus, prima decizie facuta de strategia noastra este alegerea bancnotei de 100 deoarece $ 100 	\leq 157$.  Decizia facuta are ca rezultat adaugarea valorii 1 la solutia finala iar suma pe care trebuie sa o platim scade cu 100, astfel incat la pasul urmator va trebui sa o platim 57. Acest algoritm se va repeta pana cand suma va fi 0.



// exemplu desen pasi 

\section{Limitarea Strategiei Greedy}
Limitarea strategiei alese este ca exista posibilitatea ca acesta sa nu ofere solutia optima pentru toate datele de intrare.\par
De exemplu, daca multimea de bancnote este formata din $B = \{ 1,6,9 \}$ iar suma care trebuie platita este $s = 12$, solutia optima generata de strategia aleasa va fi $\{3,0,1\}$. Cu toate acestea, solutia cu adevarat optima va fi $\{0,2,0\}$. 
\\exemplu

Motivul pentru care se produce aceasta "eroare" este ca solutia este construita pas cu pas. La fiecare pas, se alege bancnota care va crea cel mai mic cost pentru solutia curenta chiar daca in viitor ar exista o alta solutie cu cost mai mic.












