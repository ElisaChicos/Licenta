\chapter{Problema Bancnotelor}

\section{Ce este Problema Bancnotelor?}

Problema bancnotelor este o problemă de minimizare de care ne lovim zilnic. Știind că avem o  mulțime de bancnote $ B = \{ b_1, b_2, b_3\}$ și o suma $S$ de plătit, rezultatul problemei va fi mulțimea de bancnote de cardinal minim pe care o putem folosi pentru a plăti suma $S$.

\subsection{Formularea problemei}

În limbaj natural, Problema Bancnotelor poate fi formulată astfel:
\par
\vspace{0.5cm}
\fbox{\parbox{5.5in}
	{ Se dă o mulțime de bancnote $ B = \{ b_1, b_2, b_3\}$ și o suma $S$ pe care trebuie să o plătim. Se cere să se afișeze o mulțime de bancnote care trebuie să îndeplinească următoarele condiții:\par
		\setlength{\parindent}{1em}
		$\bullet$ cardinalul mulțimii să fie minim, \par
		$\bullet$ suma elementelor mulțimii să fie egata cu $S$;
}}
\par
\vspace{0.7cm}
Formularea computațională a problemei:\par
\vspace{0.5cm}
\fbox{\parbox{5.5in}
	{
		\setlength{\parindent}{0em}
		Input: \par
		\setlength{\parindent}{2em}
		$\bullet$ un număr natural n - suma care trebuie plătită,
		\par 
		\setlength{\parindent}{0em}
		\par
		Output:\par
		\setlength{\parindent}{2em}
		$\bullet$ numerele $n_{500}, n_{200}, n_{100}, n_{50}, n_{20}, n_{10}, n_5, n_1$ ($n_i$ - numărul de \par bancnote i folosite), astfel încât $ \sum_{i \in \{500,200,100,50,20,10,5,1 \}} n_i$ să fie minimă \par și $ n =  \sum_{i \in \{500,200,100,50,20,10,5,1 \} } i \times n_i$.
}}
\par 
Spre exemplu, fiind date bancnotele $B = \{100,50,10,5,1\}$ și  $s = 157$ suma care trebuie plătită. \par
Soluția optimă pentru această formulare a problemei este 
$\{1,1,0,1,2\}$ deoarece $ s = 1\cdot 100+1	\cdot 50+0	\cdot10,1	\cdot5+1	\cdot2 $.


\section{Strategia Greedy aleasă}

Strategia Greedy constă în alegerea, la fiecare pas, a bancnotei cele mai mari, care este mai mică sau egală cu suma pe care trebuie să o plătim. În funcție de alegerea făcută, va trebui să scădem valoarea bancnotei alese din suma care trebuie plătită pentru a ne asigura că avansăm progresiv către obiectivul final.
\par
Folosind exemplul de mai sus, prima decizie făcută de strategia noastră este alegerea bancnotei de 100 deoarece $ 100 	\leq 157$.  Decizia făcută are ca rezultat adăugarea valorii 1 la soluția finală iar suma pe care trebuie să o plătim scade cu 100, astfel încât la pasul următor va trebui să o plătim 57. Acest algoritm se va repetă până când suma va fi 0.


\section{Limitarea Strategiei Greedy}
Limitarea strategiei alese este că există posibilitatea ca acesta să nu ofere soluția optimă pentru toate datele de intrare.\par
De exemplu, dacă mulțimea de bancnote este formată din $B = \{ 1,6,9 \}$ iar suma care trebuie plătită este $s = 12$, soluția optimă generată de strategia aleasă va fi $\{3,0,1\}$. Cu toate acestea, soluția cu adevărat optimă va fi $\{0,2,0\}$. 

Motivul pentru care se produce această "eroare" este că soluția este construită pas cu pas. La fiecare pas, se alege bancnotă care va crea cel mai mic cost pentru soluția curentă chiar dacă în viitor ar există o altă soluție cu cost mai mic.
