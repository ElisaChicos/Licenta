\chapter{Problema Bancnotelor}

\section{Ce este Problema Bancnotelor?}
%Problema bancnotelor este o problema de care ne lovim in fiecare zi. Aceasta ne spune ca avand o multime de bancnote $ B = \{ b_1, b_2, b_3\}$ si o suma $S$ de platit, algorimtul va nt

Problema bancnotelor este o problema de minimizare de care ne lovim zilnic. Stiind ca avem o  multime de bancnote $ B = \{ b_1, b_2, b_3\}$ si o suma $S$ de platit, rezultatul problemei va fi multimea de bancnote cu cardinal minim pe care o putem folosi pentru a plati suma $S$.

\subsection{Formularea problemei}

In limbaj natural, Problema Bancnotelor poate fi formulata astfel:
\par
\vspace{0.5cm}
\fbox{\parbox{5.5in}
	{ Se da o multime de bancnote $ B = \{ b_1, b_2, b_3\}$ si o suma $S$ pe care trebuie sa o platim. Se cere sa se afiseze o multime de bancnote care trebuie sa indeplineasca urmatoarele conditii:\par
	\setlength{\parindent}{1em}
	$\bullet$ cardinalul multimii sa fie minim, \par
	$\bullet$ suma elementelor multimii sa fie egata cu $S$;
}}

\section{Titlul secțiunii 2}


\section{Titlul secțiunii 3}
